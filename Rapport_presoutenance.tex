\documentclass{beamer}

\usepackage[utf8]{inputenc}
%\usepackage[T1]{fontenc}
%\usepackage[latin1]{inputenc}

\usetheme{Warsaw}

\title[Signal segmentation]{Functionnal data analysis applied to neurology}
\author{Clément Bonvoisin, Pierre Ludmann}
\institute{CMLA (ENS Cachan), Cognac-G (Paris V)}
\date{09/04/2014}

\begin{document}
\setbeamertemplate{navigation symbols}{}
\setbeamertemplate{footline}[frame number]
%\addtobeamertemplate{footline}{\hfill\insertframenumber/\inserttotalframenumber}

\begin{frame}
\titlepage
\end{frame}

\begin{frame}
\frametitle{Outline}
  \tableofcontents[hideallsubsections]
\end{frame}

\AtBeginSection[]
{
  \begin{frame}
  \tableofcontents[currentsection, hideothersubsections]
  \end{frame} 
}

\section{Familiarisation avec les données}
\subsection{Mise au format MATLAB}

\begin{frame}
\frametitle{}
Ma première page !
\insertframenumber{}
\inserttotalframenumber
\end{frame}

\subsection{Visualisation}
\begin{frame}
Et maintenant ma deuxième page !
\end{frame}



\section{Approche par fenêtres}
\subsection{Avec Fourier}

\begin{frame}
Voici ma troisième page, elle appartient à ma deuxième section ! :) 
\end{frame}

\subsection{Avec des statistiques}
\begin{frame}
Et celle là c'est la deuxième page, mais de ma deuxième section. 
\end{frame}

\section{Tests d'hypothèses}
\subsection{}
\subsection{Algorithme CUSUM}
\end{document}