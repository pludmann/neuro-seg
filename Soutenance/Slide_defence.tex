\documentclass{beamer}

\usepackage[utf8]{inputenc}
%\usepackage[T1]{fontenc}
%\usepackage[latin1]{inputenc}

\usetheme{Warsaw}

\title[Signal segmentation]{Functional data analysis applied to neurology}
\author{Clément Bonvoisin, Pierre Ludmann}
\institute{CMLA (ENS Cachan), Cognac-G (Paris V)}
\date{09/04/2014}

\graphicspath{{/home/marvin/neuro-seg/Soutenance/}}
\setbeamersize{text margin left=1.4cm}
\begin{document}
\setbeamertemplate{navigation symbols}{}
\setbeamertemplate{footline}[frame number]
%\addtobeamertemplate{footline}{\hfill\insertframenumber/\inserttotalframenumber}

\begin{frame}
\titlepage
\end{frame}

\begin{frame}
\frametitle{Plan}
  \tableofcontents[hideallsubsections]
\end{frame}

\AtBeginSection[]
{
  \begin{frame}
  \tableofcontents[currentsection, hideothersubsections]
  \end{frame} 
}

\section{Introduction}
\subsection{Segmentation de signaux}
\begin{frame}
	\frametitle{Présentation}
	\begin{itemize}
		\item[Projet] pluridisciplinaire
			\\ Médecins
			\\ Mathématiciens
		\item[Enjeux] variés
			\\ Fournir une base de donnée aux deux acteurs
			\\ Tester des modèles sur des signaux réels
			\\ Suivi des patients
			\\ Étudier les troubles de la marche
		\item[Protocole] expérimental
			\\ Placements des capteurs
			\\ Mouvements
			\\ Référentiel de travail
	\end{itemize}	
\end{frame}

\begin{frame}
	\frametitle{Exemple}
\end{frame}
\subsection{Attentes et essais}

\begin{frame}
	\frametitle{}
	\begin{itemize}
		\item[L'affichage] montre clairement les différentes séquences de l'expérience
		\item[$\Longrightarrow$] La segmentation automatique doit être rapide et précise, au moins autant qu'à l'œil
		
		\item[Etiquettes]: Idle, Start, Stop, Walk, Turn, Trash
		
		\item[Fenetres]
		
	\end{itemize}
\end{frame}

\section{Recherche de ruptures}

\subsection{Définition}

\begin{frame}

\frametitle{Formaliser la rupture}

\begin{itemize}

	\item[Signaux]: réalisations d'une variable aléatoire
	
	\item[Rupture]: passage d'une varaible aléatoire à une autre

\end{itemize}

\end{frame}

\subsection{Algorithme CUSUM}

\begin{frame}
	\frametitle{Généralités sur l'algorithme CUSUM}
	\begin{itemize}
		\item[Biblio] \emph{Detection of Abrupt Changes : Theory and Application},\\
		M. Basseville, I. V. Nikiforov (1993)
		\item[Proposé] dans \emph{Continuous inspection scheme}, E.S. Page (1954)
		\item[Comparer] l'hypothèse d'un changement à l'hypothèse stationnaire
	\end{itemize}
	\begin{equation}
		L _k =\ln \left[ \frac{\sup_{\theta_0}\left\{ \prod_{i=1}^{k-1} p_{\theta_0}(y_i) \right\} \cdot \sup_{\theta_1} \left\{ \prod_{i = k}^{N}p_{\theta_1}(y_i) \right\}}{\sup_{\tilde\theta}\left\{\prod_{i=i}^{N}p_{\tilde{\theta}}(y_i)\right\}} \right]
	\end{equation}
	\begin{itemize}
		\item[Rupture]:
	\end{itemize}
	\begin{equation}
		\max_{1 \leq k \leq N} L_k \geq h \Rightarrow t_0 = \arg \max_{1 \leq k \leq N} L_k
	\end{equation}
	\begin{itemize}
		\item[Complexité] élevée avec les bornes supérieures : besoin de simplification
	\end{itemize}
\end{frame}

\subsection{Hypothèses de travail}

\begin{frame}
	\frametitle{Le choix du gaussien}
	Signaux supposés suivre une distribution normale :
	\begin{equation}
		p_{\mu, \sigma}(y) = \frac1{\sigma\sqrt{2 \pi}} \exp \left[ -\frac12 \left( \frac{y - \mu}{\sigma} \right)^2 \right]
	\end{equation}
	\begin{itemize}
		\item[Hypothèse] forte : indépendance temporelle et spatiale
		\item[Paramètre] $\theta$ : changement brusque de la moyenne et/ou de l'écart-type du signal
		\item[Hypothèse] utile : bornes supérieures atteintes aux estimateurs
	\end{itemize}
\end{frame}

\begin{frame}
	\frametitle{Choix des paramètres - Formules correspondantes}
	Trois choix possibles :
	\vspace*{.3cm}
	\begin{itemize}
		\item[$\theta=\mu$]: (4) avec $\mu=\frac1n\sum_{i=1}^ny_i$ et $\sigma$ fixé
		\vspace*{.2cm}
		\item[$\theta=\sigma$]:  (5) avec $\mu$ fixé et $\sigma=\sqrt{\frac1n\sum_{i=1}^n(y_i-\mu)^2}$
		\vspace*{.2cm}
		\item[$\theta=(\mu,\theta)$]: (5) avec \mbox{$\mu=\frac1n\sum_{i=1}^ny_i$ et $\sigma=\sqrt{\frac1n\left[\sum_{i=1}^ny_i^2-(\sum_{i=1}^ny_i)^2\right]}$}
	\end{itemize}
	\vspace*{0.8cm}
	\begin{equation}
	\hspace{-1cm}	L_k=\frac 1{2\sigma^2}\left[(k-1)\mu_0^2+(N-k+1)\mu_1^2-N\tilde\mu^2\right]
	\end{equation}
	ou
	\begin{equation}
	\hspace{-1cm}	L_k=N\ln(\tilde\sigma)-(k-1)\ln(\sigma_0)-(N-k+1)\ln(\sigma_1)
	\end{equation}
\end{frame}

\section{Implémentations}

\subsection{Hors ligne}

\begin{frame}

	Principe :
	\vspace{.25cm}
	\\
	Ici l'$\arg\max$ est toujours une rupture
	\\
	Meilleur sous arbre de taille $n$ de l'arbre de dichotomie de hauteur $n$, avec $n$ le nombre de ruptures à faire apparaître
	\\
	\phantom{caca}
	Avantages :

	\begin{itemize}
		
		\item Pas de seuil
		
		\item Demande le nombre de ruptures : réponse à l'oeil
		
	\end{itemize}

	Inconvénients :
	
	\begin{itemize}
	
		\item Déborde du cadre théorique
		
		\item Ruptures visibles $\ne$ ruptures importantes
	
	\end{itemize}

\end{frame}

\begin{frame}
\frametitle{Exemple}
\end{frame}

\subsection{En ligne}

\begin{frame}

	Principe :
	\vspace{.25cm}
	\\
	Dévoiler progressivement le signal jusqu'à trouver une rupture
	\\
	Recommencer depuis la rupture trouvée
	\\
	S'arrêter quand tout le signal est dévoilé et que le seuil n'est pas dépassé
	\\
	\phantom{caca}
	Avantages :
	
	\begin{itemize}
	
		\item Respecte le cadre théorique
		
	\end{itemize}
	
	Inconvénients :
	
	\begin{itemize}
	
		\item Paramètre $(\mu,\sigma)$ non fonctionnel
		
		\item Seuil sensible à régler
		
	\end{itemize}
	
\end{frame}

\begin{frame}
\frametitle{Exemple}
\end{frame}

\section{Conclusion}

\begin{frame}

\begin{itemize}

	\item[Python]
	
	\item[Capteurs]
	
	\item[BDD] bien reglée
	
	\item[ML] sur les segments

\end{itemize}

\end{frame}

\end{document}